\documentclass{article}

\usepackage{fontspec}
\setmainfont[Ligatures=TeX]{Linux Libertine O}

\usepackage{natbib}% Citation support using natbib.sty
\usepackage{graphicx}
\usepackage{amsmath}
\usepackage{hyperref} 

\bibpunct[, ]{(}{)}{;}{a}{}{,}% Citation support using natbib.sty
\renewcommand\bibfont{\fontsize{10}{12}\selectfont}% Bibliography support using natbib.sty

\title{Оптимизација на производство на електрична енергија}

\date{2018-03-23}

\author{Благој Атанасовски}

\begin{document}
\begin{figure}
\includegraphics[scale=.4]{image}
\end{figure}

\maketitle
\newpage

\section{Претставување на проблемот}
Да претпоставиме дека на територијата на една земја, постојат $N$ населени места и $M$ електрани. Секое од населените места во даден момент има потрошу-вачка/потреба од $ Potrosuvacka[i] $ MWh електрична енергија каде $ 0 < i \leq N $ . Секоја од електраните има можност во даден момент да произведе максимум $MaxProiz[j]$ MWh електрична енергија (термо електрана на јаглен) каде $0 < j \leq M$. Електраните може да го контролираат производството на енергија од $0\%$ до $100\%$. За секоја од електраните имаме $MaxCena[j]$ што ја претставува цената да работи електраната во полн капацитет. Цената е линеарно зависна од капацитетот со кој работи електраната, така што на капацитет 0\% цената е 0. Едно населено место е поврзано со далековод со барем една од електраните, а максимум со сите. 

За секој од далеководите помеѓу населено место $i$ и електрана $j$ познат е коефициентот на загуба ($Zaguba[i][j]$) што го дава процентот на изгубена моќност, при што $0 \leq Zaguba[i][j] < 100 $ и кога $Zaguba[i][j] = 0$ тогаш нема далековод кој ги поврзува i и j. Органичувања кои важат се:
\begin{itemize}
\item Електраните под полн капацитет можат да ги задоволат потребите на сите населени места
\item Поставеноста на далеководите гарантира дека секогаш ќе може да се испо-рача енергија до сите населени места
\end{itemize}

За дадени:
\begin{itemize}
\item Моќност на електричните централи
\item Потрошувачка на населните места
\item Поставеност на далеководите што ги поврзуваат електраните со населените места
\item Загубата на секој од далеководите
\end{itemize}

Целта е да се најде капацитетот со кој треба да работат електраните, така што цената за производство и транспорт на електричната енергија да биде минимална.

\section{Решение}

Зададениот проблем има оптимизациска природа, со тоа што целта е да се минимизира цената за производство на електрична енергија. Цената ја дефини-раме како:
\[cena = \sum_{i=1}^{M}OdbranKapacitet[i]\cdot MaxCena[i]\]
каде $OdbranKapacitet[j]$ е капацитетот со којшто треба да работи електраната $j$ во решението. Ако дефинираме $Isporaka[i][j]$ = Процентот од енергијата генерирана во електраната $j$ испорачана до населено место $i$, тогаш важат огра-ничувањата:

\[\forall i, 1 \geq i \geq N,\]
\[\sum_{j=1}^{M} Isporaka[i][j] \cdot OdbranKapacitet[j] \cdot MaxProiz[j] \cdot Zaguba[i][j] \geq Pot[i]\]
\[\forall j, 1\geq j \geq M,\]
\[\sum_{i=1}^{N} Isporaka[i][j] \cdot OdbranKapacitet[j] \cdot MaxProiz[j] \cdot Zaguba[i][j] \leq MaxProiz[j] \]
\[ 0 \leq Isporaka[i][j] \leq 100 \]

\section{Методологија за оценување на решенијата}
Пронаоѓањето на brute-force решение со испробување на сите комбинации прет-ставува проблем со $O(100^{n \cdot n})$  комплексност, бидејќи за секоја вредност од мат-рицата $Isporaka$ има 100 вредности. 

За лесна проверка ќе бидат земени:
\begin{enumerate}
\item Тривијални примери
	\begin{enumerate}
	\item 1 електрана со n населени места. Сите се поврзани, има еднаква загуба
	\item 1 населено место со m електрани. Сите се поврзани, има еднаква загуба
	\item 1 населено место со m електрани. Некои се неповрзани, има еднаква загуба
	\end{enumerate} 
\item Едноставни примери
	\begin{enumerate}
	\item 2 населени места, 2 електрани. Сите се поврзани меѓу себе, има иста загуба
	\item 3 населени места, 3 електрани. Сите се поврзани, има различна загуба
	\item мали m и n. Постојат неповрзани населени места со електрани, има различна загуба
	\end{enumerate}
\end{enumerate}

Дополнително, ќе биде следено влијанието на:
\begin{itemize}
\item ратата на мутација
\item ратата на вркстување
\item големината на популацијата
\item бројот на генерации
\end{itemize}

врз пронајдениот минимум и брзината на конвергенција.

\subsection{1.a}
Во примерот 1.a цел на оптимизација е распределбата на електрична енергија од една електрана кон n населени места. Сите населени места се поврзани со електраната и процентот на загуба на енергија е еднаков за сите.

Влезни аргументи: 
\begin{itemize}
\item $m = 1$
\item $n = 10$
\item $Potrosuvacka = \begin{bmatrix} 10 & 20 & 50 & 100 & 30 & 40 & 100 & 5 & 10 & 20\end{bmatrix}$
\item $MaxProiz = \begin{bmatrix} 500 \end{bmatrix}$
\item $MaxCena = \begin{bmatrix} 1000 \end{bmatrix}$
\item $\forall i 1 \leq i \leq n, \forall j 1 \leq j \leq m, Zaguba[i][j] = 10$

\end{itemize}

Очекуван излез:
\begin{itemize}
\item $PredlogDistribucija = \begin{bmatrix} 3 & 5 & 12 & 23 & 7 & 9 & 23 & 2 & 3 & 5\end{bmatrix}$
\item $ OdbranKapacitet = \begin{bmatrix} 92 \end{bmatrix}$
\item $cena = 0.92 \cdot 1000 = 920 $
\end{itemize}

\subsection{1.b}
Во примерот 1.b цел на оптимизација е распределбата на електрична енергија до 1 населено со m електрани. Сите електрани се поврзани со населеното место и процентот на загуба на енергија е еднаков за сите.

Влезни аргументи: 
\begin{itemize}
\item $m = 5$
\item $n = 1$
\item $Potrosuvacka = \begin{bmatrix} 100 \end{bmatrix}$
\item $MaxProiz = \begin{bmatrix} 20 & 30 & 10 & 50 & 100\end{bmatrix}$
\item $MaxCena = \begin{bmatrix} 10 & 20 & 5 & 10 & 120\end{bmatrix}$
\item $\forall i 1 \leq i \leq n, \forall j 1 \leq j \leq m, Zaguba[i][j] = 10$

\end{itemize}

Очекуван излез:
\begin{itemize}
\item $PredlogDistribucija = \begin{bmatrix} 100 & 98 & 98 & 100 & 1 \end{bmatrix}$
\item $OdbranKapacitet = \begin{bmatrix} 100 & 98 & 98 & 100 & 1\end{bmatrix}$
\item $cena = 10 + 20 + 5 + 10 + 0.01 \cdot 120 = 46.2 $
\end{itemize}

\subsection{1.c}

Во примерот 1.c цел на оптимизација е распределбата на електрична енергија до 1 населено место а на располагање се m електрани. Некои од електраните не се поврзани со електраната и процентот на загуба е еднаков за сите.

Влезни аргументи: 
\begin{itemize}
\item $n = 1$
\item $m = 5$
\item $Potrosuvacka = \begin{bmatrix} 100 \end{bmatrix}$
\item $MaxProiz = \begin{bmatrix} 20 & 30 & 10 & 50 & 100 \end{bmatrix}$
\item $MaxCena = \begin{bmatrix} 100 & 5 & 50 & 10 & 120 \end{bmatrix}$
\item $Zaguba[1][1] = 10, Zaguba[1][3] = 10, Zaguba[1][5] = 10$
\item $Zaguba[1][2] = 0, Zaguba[1][4] = 0$ т.е. нема довод
\end{itemize}

Очекуван излез:
\begin{itemize}
\item $PredlogDistribucija = \begin{bmatrix}  10 & 0 & 100 & 0 & 100 \end{bmatrix}$
\item $OdbranKapacitet = \begin{bmatrix} 10 & 0 & 100 & 0 & 100\end{bmatrix}$
\item $cena = 0.1 \cdot 100 + 0 + 50 + 0 + 100 = 234 $
\end{itemize}

\subsection{2.a}
Во примерот 2.a цел на оптимизација е распределба на електрична енергија до 2 населени места од 2 електрани, ако и двете електрани се поврзани до двете населени места и имаат иста загуба. 

Влезни аргументи:
\begin{itemize}
\item $n = 2$
\item $m = 2$
\item $Potrosuvacka = \begin{bmatrix} 150 & 200\end{bmatrix}$
\item $MaxProiz = \begin{bmatrix} 250 & 200 \end{bmatrix}$
\item $MaxCena = \begin{bmatrix} 120 & 50 \end{bmatrix}$
\item $Zaguba = \begin{bmatrix} 10 & 10\\ 10 & 10 \end{bmatrix}$
\end{itemize}

Очекуван излез:
\begin{itemize}
\item $PredlogDistribucija = \begin{bmatrix} 0 & 84 \\ 69 & 26 \end{bmatrix}$
\item $OdbranKapacitet = \begin{bmatrix} 69 & 100 \end{bmatrix}$
\item $cena = 0.69 * 120 + 1 * 50 = 132.8$
\end{itemize}

\subsection{2.b}
Во примерот 2.b цел на оптимизација е распределба на електрична енергија до 4 населени места од 4 електрани, ако секое од населените места има довод од само 2 електрани. 

Влезни аргументи:
\begin{itemize}
\item $n = 4$
\item $m = 4$
\item $Potrosuvacka = \begin{bmatrix} 150 & 200 & 500 & 300 \end{bmatrix}$
\item $MaxProiz = \begin{bmatrix} 250 & 300 & 1000 & 400\end{bmatrix}$
\item $MaxCena = \begin{bmatrix} 120 & 50 & 400 & 100 \end{bmatrix}$
\item $Zaguba = \begin{bmatrix} 10 & 10 & 0 & 0 \\ 0 & 10 & 10 & 0 \\ 0 & 0 & 10 & 10 \\ 10 & 0 & 0 & 10 \end{bmatrix}$
\end{itemize}
Очекуван излез:
\begin{itemize}
\item $PredlogDistribucija = \begin{bmatrix} 0 & 56 & 0 & 0 \\ 0 & 41 & 10 & 0 \\ 0 & 0 & 50 & 14 \\ 0 & 0 & 0 & 84 \end{bmatrix}$
\item $OdbranKapacitet = \begin{bmatrix} 0 & 97 & 60 & 98 \end{bmatrix}$
\item $cena = 0 \cdot 120 + 0.97 \cdot 50 + 0.6 \cdot 400 + 0.98 \cdot 100  = 386.5$
\end{itemize}


\subsection{2.c}
Во примерот 2.c цел на оптимизација е распределба на електрична енергија до 7 населени места од 4 електрани, ако секое од населените места има довод од барем 2 електрани. 

Влезни аргументи:
\begin{itemize}
\item $n = 7$
\item $m = 4$
\item $Potrosuvacka = \begin{bmatrix} 100 & 100 & 100 & 100 & 100 & 100 & 100\end{bmatrix}$
\item $MaxProiz = \begin{bmatrix} 300 & 150 & 300 & 150\end{bmatrix}$
\item $MaxCena = \begin{bmatrix} 100 & 80 & 300 & 150 \end{bmatrix}$
\item $Zaguba = \begin{bmatrix} 0 & 10 & 10 & 10 \\ 10 & 0 & 10 & 10 \\ 10 & 10 & 0 & 10 \\ 10 & 10 & 10 & 0 \\ 0 & 0 & 5 & 5 \\ 5 & 5 & 0 & 0 \\ 10 & 10 & 10 & 10\end{bmatrix}$
\end{itemize}
Очекуван излез:
\begin{itemize}
\item $PredlogDistribucija = \begin{bmatrix} 0 & 0 & 38 & 0 \\ 0 & 0 & 38 & 0 \\ 0 & 0 & 0 & 75 \\ 38 & 0 & 0 & 0 \\ 0 & 0 & 24 & 23 \\ 36 & 0 & 0 & 0 \\ 26 & 21 & 0 & 2 \end{bmatrix}$
\item $OdbranKapacitet = \begin{bmatrix} 100 & 21 & 100 & 100 \end{bmatrix}$
\item $cena = 1 * 100 + 0.21 * 80 + 1 * 60 + 1 * 40 = 216.8$
\end{itemize}


\section{Имплементација на решението}

Решенијата се имплементирани во програмскиот јазик Python 2.7 со користење на библиотеката PyEvolve. Оваа блиблиотека содржи готови имплементации на еволутивни алгоритми и дозволува нивно извршување со едноставно мену-вање на параметрите. Во неа постојат веќе креирани едноставни репрезентации на хромозоми (1д низа, 2д низа, бинарен стринг) и оператори за нив. Исто така овозможува и креирање на свои оператори.

\subsection{Верзија 1: Наивна имплементација}

Првата имплементација претставува наивен обид да се претстават ограничува-њата дадени во описот на проблемот само преку функцијата на целта. Односно, доколку некое од ограничувањата не е исполнето, тогаш цената се „казнува“ неколкукратно. 

\subsubsection{Претставување на хромозомите}

Хромозомите ги претставуваме како дводимензионална матрица: 
\[Predlog\ Distribucija[i][j], 0 < i \leq N, 0 < j < M \]
каде елемент на позиција $[i][j]$ го претставува делот од производството на елек-траната $j$ (изразено во проценти) кој се доставува на населеното место 
\[i \forall i, \forall j, 0 \leq PredlogDistribucija[i][j] \leq 100\] Од иницијалнито претставување на решението 
\[ \forall i, OdbranKapacitet[i] = \sum_{j=1}^{N} PredlogDistribucija[i][j]\]

На иницијализација вредностите на секое место во матрицата се поставуваат по случаен избор со рамномерна распределба.

\subsubsection{Мутација}

Операторот за мутација како аргумент прима еден хромозом и ратата на мута-ција со која е стартуван алгоритмот. Потоа, зависно од ратата на мутација, бира колку од вредностите на матрицата на хромозомот ќе ги мутира т.е колку од нив ќе бидат поставени на случајна вредност помеѓу 0 и 100.
 
\subsubsection{Вкрстување}

Операторот за вкрстувањето како аргумент добива два хромозома и за секој од елементите во матрицата на хромозомите на иста позиција постои 50\% шанса да бидат заменети меѓу двете единки. 
 
\subsubsection{Функција на цел}

Функцијата на цел на влез прима еден хромозом и ја пресметува цената за производството на електричната енергија. Најпрво се пресметува колку изнесу-ва цената според избраниот капацитет на секоја од електраните 
\[ cena = \sum_{j=1}^{M} (maxCena[j] \cdot MaxProiz[j] \cdot  \sum_{i=1}^{N} OdbranKapacitet[i][j] )\]
потоа, доколку не е исполнето некое од ограничувањата, цената се казнува со множење. Притоа:
\begin{itemize}
\item доколку електрана дистрибуира енергија до населено место до кое нема далековод ($Zaguba[i][j] = 0$ но $OdbranKapacitet[i][j] \neq 0$), цената се множи со производот на цената на најскапата електрана и големината на прекорачувањето  
\item доколку електрана треба да произведе повеќе енергија од нејзиниот мак-симален капацитет ($OdbranKapacitet[i][j] > 100$), цената се множи со производот на цената на најскапата електрана и големината на премину-вањето 
\item доколку населено место добива помалку енергија од својата побарувачка \[ \forall i, \sum_{j=1}^{M} MaxProiz[j] \cdot OdbranKapacitet[i][j] \cdot Zaguba[i][j] < Potrosuvacka[i]\] цената се множи со производот на цената на најскапата електрана и недо-статокот на енергија 
\end{itemize}

Алгоритмот се обидува да ја минимизира цената. Казните на цената зависат од влезните податоци (цена на најскапа електрана). Казните се променливи и во однос на големината на грешката, преминувањето на капацитетот на една електрана за 1\% се казнува пропорционално помалку во однос на преминување-то на капацитетот за 20\% или 30\%.

\subsubsection{Резултати}
 
Резултатите ќе ги прикажеме во табела за секоја од категоријата на решенија, споредени со очекуваните резултати. Во табелата дадена во оваа секција ги прикажуваме резултатите добиени за 10000 генерации со 1000 единки во попу-лацијата. Притоа, алгоритмот работи со рата на мутација 0.5 и рата на вкрс-тување 0.5. Во продолжение ќе ги дискутираме добиените резултати и ќе наве-деме какво влијание имала промената на разни параметри на алгоритмот. 

\begin{table}[h!]
\centering
\begin{tabular}{||c c c||} 
 \hline
 Проблем & Добиено решение & Очекувано решение \\ [0.5ex]
 \hline\hline
 1.a & 920 & 920\\ 
 1.b & 46.9 & 46.2 \\
 1.c & 176 & 160 \\
 2.a & 141.2 & 132.8 \\
 2.b & 456 & 386.5 \\
 2.c & N/A & 216.8 \\ [1ex] 
 \hline
\end{tabular}
\caption{Споредба на добиени резултати со очекуваните, 1000 единки во 10000 генерации}
\label{table_rez_ver1}
\end{table}

Од табела \ref{table_rez_ver1} може да се забележи дека за најтривијалните проблеми наивно-то генетско решение дава добри резултати. Но, со усложнување на проблемот, воведувањето на повеќе електрани кои дистрибуираат кон повеќе населени места со променливи загуби, веќе резултираат со значителни отстапувања од очекуваните решенија. Во примерите 2.b и 2.c очекуваните решенија исто така не се оптимални, туку се добиени на алчен пристап и служат само за споредба. И покрај тоа, првата верзија на генетскиот алгоритам не дава решенија блиски до нив. За најсложениот пример во 2.c генетскиот алгоритам не успеа да даде решение кое не ги прекршува ниту едно од дадените ограничувања (дури и по 15000 генерации со 1000 единки). 

Дополнително се тестира и влијанието на промената на: ратата на мутација, ратата на вкрстување, бројот на генерации и бројот на единки. Од табелата во документот со решенија референциран во секција \ref{sec_apendix} може да се забележи дека за оваа имплементација на алгоритмот, рата на мутација под 0.4 и над 0.6 влијае забележително негативно на резултатите, освен во наједноставните случаи. Ратата на вкрстување има многу мало влијание на резултатите. За примерите од групата 1, бројот на единки нема влијание и доволни се 100 единки за да се добие добар резултат. Но, за примерите од втората група (особе-но 2.b), тестовите со помалку од 1000 единки не даваат резултати кои ги задово-луваат ограничувањата. На сличен начин влијае и бројот на генерации. Кај тривијалните примери, доволни беа и 1000 генерации, додека пак кај 2.c приме-рот ниту 15000 генерации не даваат задоволително решение.

\subsection{Верзија 2: Ограничено модифицирање на хромозомите}

Во втората верзија на алгоритмот се ограничува начинот на составување на хромозомите, со тоа што уште при иницијализација на единките се запазуваат некои од ограничувањата. Исто така при вкрстувањето се внимава потомокот да ги задоволува истите ограничувања.

\subsubsection{Претставување на хромозомите}

Хромозомите ги претставуваме потполно исто како во Верзија 1, со тоа што во оваа варијанта при иницијализирањето на еден хромозом се зема во предвид матри-цата на загуба. Поради тоа што оваа матрица е позната на почеток на алгоритмот и ни кажува меѓу кои населени места и електрани не треба да има никаква дистрибуција, знаеме дека на тие места во матрицата $PredlogDistribucija$ во решението треба да има нули. Односно, при иницијализација, наместо случајна вредност се поставува нула. На овој начин едно од ограничувањата е задоволено од сите единки во популацијата

\subsubsection{Мутација}

Операторот за мутација е истиот како во Верзија 1, но доколку според матрицата $Zaguba$ не треба да има дистрибуција меѓу одредено населено место и електрана, тогаш вредноста во матрицата $PredlogDistribucija$ на тоа место ќе остане 0.
 
\subsubsection{Вкрстување}

Операторот за вкрстувањето добива како аргумент 2 хромозома и за секој од елементите во матрицата на хромозомите на иста позиција постои 50\% шанса да бидат заменети меѓу двете единки. Но, ако според матрицата $Zaguba$ нема дистрибу-ција меѓу два елементи, тогаш на таа позиција останува 0.
 
\subsubsection{Функција на цел}

Функцијата на цел останува иста како во верзија 1.

\subsubsection{Резултати}
 
Резултатите ќе бидат прикажани во табела за секоја од категоријата на решенија, споредени со очекуваните резултати. Во табелата дадена во оваа секција се прикажани резултатите добиени за 10000 генерации со 1000 единки во попула-цијата. Притоа алгоритмот работи со рата на мутација 0.5 и рата на вкрстување 0.5. Потоа ќе се дискутираат добиените резултати и ќе биде наведено какво влијание има промената на разни параметри на алгоритмот. 

\begin{table}[h!]
\centering
\begin{tabular}{||c c c||} 
 \hline
 Проблем & Добиено решение & Очекувано решение \\ [0.5ex]
 \hline\hline
 1.a & 920 & 920\\ 
 1.b & 46.9 & 46.2 \\
 1.c & 176 & 160 \\
 2.a & 141.2 & 132.8 \\
 2.b & 385.7 & 386.5 \\
 2.c & 241.6 & 216.8 \\ [1ex] 
 \hline
\end{tabular}
\caption{Споредба на добиени резултати со очекуваните, 1000 единки во 10000 генерации}
\label{table_rez_ver2}
\end{table}

Од табела \ref{table_rez_ver2} може да се забележи дека резултатите од претходната верзија на алгоритмот значително се подобрени, посебно за примерот 2.b и 2.c. Кај примерот 2.b се добива подобро решение и од очекуваното, т.е. решението кое беше пресметано на рака со алчен пристап на обид и грешка. Кај примерот 2.c со првата варијанта на алгоритмот не успеавме да добиеме решение кое ќе ги задоволува сите ограничувања, со ова варијанта се добива решение. Предлог дистрибуцијата за примерот 2.b е 
\[PredlogDistribucija = \begin{bmatrix} 0 & 56 & 0 & 0 \\ 0 & 41 & 10 & 0 \\ 0 & 0 & 49 & 17 \\ 1 & 0 & 0 & 83 \end{bmatrix}\]

Дополнително беше тестирано и влијанието на: намалена рата на мутација, зголемена и намалена рата на вкрстување и зголемен број на генерации. Од табелата со сите резултати од тестовите референцирана во секција \ref{sec_apendix} може да се забележи дека за оваа имплементација на алгоритмот:
\begin{itemize}
\item користење на повеќе од 10000 итерации дава погодни резултати само за примерот 2.c, кој е најсложениот од примерите, (пронајдено беше решение со цена од 228.4 за разлика од претходното 241.6) и може да се извлече хевристика која алгоритмот ќе го конфигурира колку итерации ќе работи во зависност од должината на влезните податоци
\item зголемена рата на вкрстување не дава подобри резултати
\item намалена рата на вкрстување дава малку подобри резултати само за приме-рот 2.c 
\end{itemize}

\section{Заклучок}
Градењето на генетски алгоритам за проблемот претставен во овој текст се покажа како доволно добра опција. Просторот на решенија кој треба да се пребара значи-телно се намалува во споредба со brute-force решенија. Својствата на генетскиот алгоритам овозможуваат пребарува-њето да биде добро насочено. Со втората верзија на алгоритмот се помогна тој да конвергира во решение кое ги задоволува сите органичувања многу порано и кај сите тест примери. Можноста некои од ограничувањата да бидат запазени од самите генетски опе-ратори резултираше со доста поволност за работата на алгоритмот и неговата способност да решава посложени проблеми. За посложени влезни податоци, алгоритмот работи и дава резултати, но не скалира многу добро во поглед на време на извршување.
Направени беа и тестови каде функцијата на цел ги оценуваше единките кои  нарушуваат барем едно ограничување со максимална цена, но резултатите беа многу лоши.
 
\section{Дополнителни документи}
\label{sec_apendix}
Целосната имплементација е прикачена во јавен репозиториум на веб страница-та GitHub на \href{https://github.com/atanasovskib/power_distribution_optimization}{https://github.com/atanasovskib/power\_distribution\_optimization}. Тука воедно се наоѓа и документацијата на проектот. Сите резултати од тестовите се наоѓаат во датотеката \href{https://github.com/atanasovskib/power_distribution_optimization/blob/master/docs_mk/power_distribution_results.xlsx}{power\_distribution\_results.xslx}
%
%\bibliographystyle{tfcad}
%\bibliography{qqp}

\end{document}